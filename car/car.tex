\documentclass[12pt]{article}
\usepackage{fullpage}
\usepackage{enumitem}
\usepackage{amsmath}
\usepackage{amssymb}
\usepackage{graphicx}
\usepackage{bm}
\usepackage{hyperref}
\usepackage{bbm}
\usepackage{verbatim}

\begin{document}

\begin{center}
{\Large CS221 Fall 2018 Homework 7}

\begin{tabular}{rl}
SUNet ID: & 05794739 \\
Name: & Luis Perez \\
Collaborators: &
\end{tabular}
\end{center}

By turning in this assignment, I agree by the Stanford honor code and declare
that all of this is my own work.

\section*{Problem 1}

\begin{enumerate}[label=(\alph*)]
  \item We compute $\mathbb{P}(C_2 = 1 \mid D_2 = 0)$. We note that by the factor graph, we have the following:
    \begin{align*}
      \mathbb{P}(C_2 = c_2 \mid D_2 = 0) &\propto p(D_2 = 0 \mid C_2 = c_2)\sum_{c_1 \in \{0,1 \}} p(C_2 = c_2 \mid C_1 = c_1)p(C_1 = c_1) \\
      &\propto p(D_2 = 0 \mid C_2 = c_2)\sum_{c_1 \in \{0,1 \}} p(C_2 = c_2 \mid C_1 = c_1) \tag{$p(C_1 = c_1) = 0.5$, which we can drop since it's just a proportionality constant} \\
      &\propto p(D_2 = 0 \mid C_2 = c_2) \tag{$\forall c_2, \sum_{c_1} p(c_2 \mid c_1)=1$ is a valid probability distribution} \\
      &\propto p(D_2 = 0 \mid C_2 = c_2)
    \end{align*}
    Note that $p(d_2 \mid c_2)$ is a valid probability distribution, so the proportionality constant is $1$. Then we have:
    \begin{align*}
      \mathbb{P}(C_2 = 0 \mid D_2 = 0) = p(D_2 = 0 \mid C_2 = 0) = 1- \eta \\
      \mathbb{P}(C_2 = 1 \mid D_2 = 0) = p(D_2 = 0 \mid C_2 = 1) =  \eta
    \end{align*}
  \item We compute $\mathbb{P}(C_2 = 1 \mid D_2 = 0, D_3 = 1)$. We note that by the factor graph, we have the following:
   \begin{align*}
    &\mathbb{P}(C_2 = c_2 \mid D_2 = 0, D_3 = 1)\\ &\propto \mathbb{P}(D_3 = 1 \mid D_2 = 0, C_2 = c_2)p(D_2 = 0 \mid C_2 = c)p(c_2 = c_2) \tag{Bayes' Rule} \\
    &\propto \mathbb{P}(D_3 = 1 \mid D_2 = 0, C_2 = c_2)p(D_2 = 0 \mid C_2 = c) \tag{$\mathbb{P}(C_2 = c_2) = \frac{1}{2}$} \\
    &\propto \mathbb{P}(D_3 = 1 \mid C_2 = c_2)p(D_2 = 0 \mid C_2 = c) \tag{$D_3 \perp D_2 \mid C_2$} \\
    &\propto \left[\sum_{c_3 \in \{0,1\}} p(D_3 = 1 \mid C_3 = c_3, C_2 = c_2)p(C_3 = c_3 \mid C_2 = c_2)\right]p(D_2 = 0 \mid C_2 = c) \tag{LOTP} \\
    &\propto \left[\sum_{c_3 \in \{0,1\}} p(D_3 = 1 \mid C_3 = c_3)p(C_3 = c_3 \mid C_2 = c_2)\right]p(D_2 = 0 \mid C_2 = c) \tag{$D_3 \perp C_2 \mid C_3$}
    \end{align*}
    From the above and given the previous result, we compute directly the requested values. We just plug-in and lookup the corresponding conditional distributions:
    \begin{align*}
      \mathbb{P}(C_2 = 0 \mid D_2 = 0, D_3 = 1) &\propto [\eta (1-\epsilon) + (1-\eta)\epsilon](1-\eta) \\
      \mathbb{P}(C_2 = 1 \mid D_2 = 0, D_3 = 1) &\propto  [\eta\epsilon + (1-\eta)(1-\epsilon)]\eta
    \end{align*}
    From the abive and the fact that we must have a valid distributions, we arrive at the following:
    $$
      \mathbb{P}(C_2 = 1 \mid D_2 = 0, D_3 = 1) =  \frac{[\eta\epsilon + (1-\eta)(1-\epsilon)]\eta}{[\eta\epsilon + (1-\eta)(1-\epsilon)]\eta + [\eta (1-\epsilon) + (1-\eta)\epsilon](1-\eta)}
    $$
  \item We now compute the probabilities requested where $\epsilon = 0.1$ and $\eta = 0.2$.
  \begin{enumerate}[label=(\roman*)]
    \item We have:
    $$
      \mathbb{P}(C_2 = 1 \mid D_2 = 0) = \eta = 0.2
    $$
    and 
    $$
      \mathbb{P}(C_2 = 1 \mid D_2 = 0, D_3 =1) = \frac{[0.2(0.1) + (0.8)(0.9)]0.2}{[0.2(0.1) + (0.8)(0.9)]0.2 + [0.2 (0.9) + (0.8)(0.1)](0.8)} = 0.4157
    $$
    \item By adding the second sensor reading, the probability that the car was at $C_1 = 1$ increased. This change makes sense, since the second sensor reading $D_3 = 1$ sees that car at position $1$. Given that our sensor error rate is low ($\eta = 0.2$), it's likely that the car is at position $1$, according to this second reading. Furthermore, since cars change positions with low probability $\epsilon = 0.2$, the probability that $C_2 = 1$ must increase. However, note that the probability is still less than $0.5$. This is because our original sensor reading of $D_2 = 0$ has less room for error, so it's still more likely that the positions at $t = 2$ was $0$ and that the car simply moved. This is because of our relatively low $\eta$ value.

    \item We would have to set $\epsilon = 0.5$. This is because with $\epsilon = 0.5$, at each time-step, the car has equal probabiilty of staying at the current position or alternating to a new one. As such, the sensor readings then become indendent. An additional reading of $D_3 = 1$ gives no further information about the position at $t = 2$, since the car was equally likely to have stayed or swapped positions.
  \end{enumerate}
\end{enumerate}
\end{document}