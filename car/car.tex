\documentclass[12pt]{article}
\usepackage{fullpage}
\usepackage{enumitem}
\usepackage{amsmath}
\usepackage{amssymb}
\usepackage{graphicx}
\usepackage{bm}
\usepackage{hyperref}
\usepackage{bbm}
\usepackage{verbatim}

\begin{document}

\begin{center}
{\Large CS221 Fall 2018 Homework 7}

\begin{tabular}{rl}
SUNet ID: & 05794739 \\
Name: & Luis Perez \\
Collaborators: &
\end{tabular}
\end{center}

By turning in this assignment, I agree by the Stanford honor code and declare
that all of this is my own work.

\section*{Problem 1}

\begin{enumerate}[label=(\alph*)]
  \item We compute $\mathbb{P}(C_2 = 1 \mid D_2 = 0)$. We note that by the factor graph, we have the following:
    \begin{align*}
      \mathbb{P}(C_2 = c_2 \mid D_2 = 0) &\propto p(D_2 = 0 \mid C_2 = c_2)\sum_{c_1 \in \{0,1 \}} p(C_2 = c_2 \mid C_1 = c_1)p(C_1 = c_1) \\
      &\propto p(D_2 = 0 \mid C_2 = c_2)\sum_{c_1 \in \{0,1 \}} p(C_2 = c_2 \mid C_1 = c_1) \tag{$p(C_1 = c_1) = 0.5$, which we can drop since it's just a proportionality constant} \\
      &\propto p(D_2 = 0 \mid C_2 = c_2) \tag{$\forall c_2, \sum_{c_1} p(c_2 \mid c_1)=1$ is a valid probability distribution} \\
      &\propto p(D_2 = 0 \mid C_2 = c_2)
    \end{align*}
    Note that $p(d_2 \mid c_2)$ is a valid probability distribution, so the proportionality constant is $1$. Then we have:
    \begin{align*}
      \mathbb{P}(C_2 = 0 \mid D_2 = 0) = p(D_2 = 0 \mid C_2 = 0) = \eta \\
      \mathbb{P}(C_2 = 1 \mid D_2 = 0) = p(D_2 = 0 \mid C_2 = 1) = 1 - \eta
    \end{align*}
  \item We compute $\mathbb{P}(C_2 = 1 \mid D_2 = 0, D_3 = 1)$. We note that by the factor graph, we have the following:
   \begin{align*}
    \mathbb{P}(C_2 = c_2 \mid D_2 = 0, D_3 = 1) &\propto \mathbb{P}(D_2 = 0 \mid D_3 = 1, C_2 = c_2)\mathbb{P}(D_3 = 1 \mid C_2 = c_2)\mathbb{P}(C_2 = c_2) \tag{Bayes' Rule} \\
    &\propto p(D_2 = 0 \mid C_2 = c_2)\mathbb{P}(D_3 = 1 \mid C_2 = c_2)\mathbb{P}(C_2 = c_2) \tag{$D_2 \perp D_3 \mid C_2$} \\
    &\propto p(D_2 = 0 \mid C_2 = c_2)\sum_{c_3 \in \{0,1\}} p(D_3 = 1 \mid C_3 = c_3) \\
    &\sum_{c_2' \in \{0,1\}} p(C_3 = c_3 \mid C_2' = c_2')\sum_{c_1' \in \{0,1\}} p(C_2' = c_2' \mid C_1' = c_1')p(c_1') \\
    &\sum_{c_1 \in \{0,1\}} p(C_2 = c_2 \mid C_1 = c_1)p(c_1) \\
    &= 
   \end{align*}
\end{enumerate}
\end{document}