\documentclass[12pt]{article}
\usepackage{fullpage}
\usepackage{enumitem}
\usepackage{amsmath}
\usepackage{amssymb}
\usepackage{graphicx}
\usepackage{bm}
\usepackage{hyperref}

\begin{document}

\begin{center}
{\Large CS221 Fall 2018 Homework 4}

\begin{tabular}{rl}
SUNet ID: & 05794739 \\
Name: & Luis Perez \\
Collaborators: &
\end{tabular}
\end{center}

By turning in this assignment, I agree by the Stanford honor code and declare
that all of this is my own work.

\section*{Problem 1}

\begin{enumerate}[label=(\alph*)]
  \item We give the value for each iteration. We note that $V^0_{\text{opt}}(s) = 0$ to start out. We also note that since for $s_t \in \{-2, 2\}$ we are at a terminal state, we'll have $V_{\text{opt}}(s_t) = 0$ for all iterations.

  \begin{enumerate}
    \item After iteration $0$, we'll have:
    \begin{align*}
      V^0_{\text{opt}}(-1) &= 0 \\
      V^0_{\text{opt}}(0) &= 0 \\
      V^0_{\text{opt}}(1) &= 0
    \end{align*}
    \item After the first iteration, we'll have the following values:
    \begin{align*}
      V^1_{\text{opt}}(-1) &= \max_{a \in \{-1,1\}}\{0.8[20 + V^0_{\text{opt}}(-2)] + 0.2[-5 + V^0_{\text{opt}}(0)], 0.7[20 + V^0_{\text{opt}}(-2)] + 0.3[-5 V^0_{\text{opt}}(0)] \} \\ &= 15 \\
      V^1_{\text{opt}}(0) &= \max_{a \in \{-1,1\}}\{0.8[-5 + V^0_{\text{opt}}(-1)] + 0.2[-5 + V^0_{\text{opt}}(1)], 0.7[-5 + V^0_{\text{opt}}(-1)] + 0.3[-5 + V^0_{\text{opt}}(1)] \}\\  &= -5 \\
      V^1_{\text{opt}}(1) &= \max_{a \in \{-1,1\}}\{0.8[-5+ V^0_{\text{opt}}(0)] + 0.2[100 + V^0_{\text{opt}}(2)], 0.7[-5 + V^0_{\text{opt}}(0)] + 0.3[100 + V^0_{\text{opt}}(2) \} \\ &= 26.5
    \end{align*}
    \item Finally, after the second iteration, we'll have:
    \begin{align*}
      V^2_{\text{opt}}(-1) &= \max_{a \in \{ -1, 1 \}}\{0.8[20 + V^1_{\text{opt}}(-2)] + 0.2[-5 + V^1_{\text{opt}}(0)], 0.7[20 + V^1_{\text{opt}}(-2)] + 0.3[-5 + V^1_{\text{opt}}(0)] \}\\
       &= 14 \\
      V^2_{\text{opt}}(0) &= \max_{a \in \{ -1, 1 \}}\{0.8[-5 + V^1_{\text{opt}}(-1)] + 0.2[-5 + V^1_{\text{opt}}(1)], 0.7[-5 + V^1_{\text{opt}}(-1)] + 0.3[-5 + V^1_{\text{opt}}(1)] \} \\
      &= 13.45 \\
      V^2_{\text{opt}}(1) &= \max_{a \in \{ -1, 1 \}}\{0.8[-5 + V^1_{\text{opt}}(0)] + 0.2[100 + V^1_{\text{opt}}(2)], 0.7[-5 + V^1_{\text{opt}}(0)] + 0.3[100 + V^1_{\text{opt}}(2) \}\\
       &= 23
    \end{align*}
  \end{enumerate}
  \item We interpret this question as asking for the resulting optimal policy for non-terminal states after two iterations. In that case, we have:
  \begin{align*}
      \pi^2_{\text{opt}}(-1) &= \arg\max_{a \in \{ -1, 1 \}}\{0.8[20 + V^1_{\text{opt}}(-2)] + 0.2[-5 + V^1_{\text{opt}}(0)], 0.7[20 + V^1_{\text{opt}}(-2)] + 0.3[-5 + V^1_{\text{opt}}(0)] \}\\
       &= -1 \\
      \pi^2_{\text{opt}}(0) &= \arg\max_{a \in \{ -1, 1 \}}\{0.8[-5 + V^1_{\text{opt}}(-1)] + 0.2[-5 + V^1_{\text{opt}}(1)], 0.7[-5 + V^1_{\text{opt}}(-1)] + 0.3[-5 + V^1_{\text{opt}}(1)] \} \\
      &= 1 \\
      \pi^2_{\text{opt}}(1) &= \arg\max_{a \in \{ -1, 1 \}}\{0.8[-5 + V^1_{\text{opt}}(0)] + 0.2[100 + V^1_{\text{opt}}(2)], 0.7[-5 + V^1_{\text{opt}}(0)] + 0.3[100 + V^1_{\text{opt}}(2) \}\\
       &= 1
    \end{align*}
\end{enumerate}

\section*{Problem 2}

\begin{enumerate}[label=(\alph*)]
  \item It is not always the case that $V_1(s_{\text{start}}) \geq V_2(s_{\text{start}})$. For a counter-examples, see ``submission.py''.
  \item The algorithm is rather straight-forward in the case where we have an acyclic MDP.
  \begin{itemize}
    \item The first-step in the algorithm is to topologically sort the graph. It is well-known that for a DAG, a topological sorting is possible and can be computed by a modified version of DFS in linear time \footnote{\url{https://en.wikipedia.org/wiki/Topological_sorting\#Depth-first_search}}
    \item Once we have this topological sorting of the states, we process each state in reverse-topological order and compute
    $$
      V(s) = \max_{a \in A} \left\{ \sum_{s' \in \text{Succ}(s,a)}T(s,a,s' )[R(s,a,s') + V(s')] \right\}
    $$ directly for each such state.
    \item After this single pass over the states, we return the resulting value function.
  \end{itemize}
  We claim that the computed $V(s) = V_{\text{opt}}(s)$ (ie, in this single pass, we've computed the optimal value function). To undertand why, we must recall that a topological sorting is one such that for every edge transition $(s, a, s')$, from $s$ to $s'$, $s$ comes before $s'$ in the ordering. In our algorithm above, we processed these states in reverse order (ie, we calculate the value of $s'$ before we compute the value of $s$). More formally, consider all terminal states (ie, states with no successors). These states are processed first by our algorithm, given us the base case:
  \begin{align*}
    V(s') = 0 = V_\text{opt}(s') \tag{for all terminal states $s'$}
  \end{align*}
  Now let us assume $V(s') = V_\text{opt}(s')$ for all $s'$ which our algorithm has already processed (ie, if our algorithm is processing states $s$, then the above holds true for all states $s'$ which fall after $s$ in the toplogical sort). Consider the processing of state $s$. For this state, our algorithm will compute:
  \begin{align*}
    V(s) &= \max_{a \in A} \left\{ \sum_{s' \in \text{Succ}(s,a)}T(s,a,s' )[R(s,a,s') + V(s')] \right\} \tag{definition of our algorithm} \\
    &= \max_{a \in A} \left\{ \sum_{s' \in \text{Succ}(s,a)}T(s,a,s' )[R(s,a,s') + V_{\text{opt}}(s')] \right\} \tag{all $s'$ are descendants of $s$, and therefore, by the inductive hypothesis, we have $V(s') = V_\text{opt}(s')$} \\
    &= V_{\text{opt}}(s) \tag{definition of $V_\text{opt}$}
  \end{align*}
  \item Following the hint, the solution to this problem is essentially given to us in lecture. The problem already provides States', Actions'(s), and $\gamma'$. As per Percy's lecture notes, we define the transition probabilities and rewards as follows:
  \begin{align*}
    T'(s, a, s') &=
      \begin{cases} 
        (1-\gamma) & s' = o \\
        \gamma T(s, a, s') & \text{otherwise}
     \end{cases} \\
    R'(s, a, s') &= 
      \begin{cases} 
        0 & s' = o \\
        R(s, a, s') & \text{otherwise}
      \end{cases}
  \end{align*}
  Informally, with probability $(1-\gamma)$ every state can now end in a terminal state with (receiving $0$ reward). All other transitions probabilities are discounted by $\gamma$. We claim that $V_{\text{opt}}(s) = V'_{\text{opt}}$ for all $s \in \text{States}$. We can prove this directly. First, let's recall that if $V_{\text{opt}}(s)$ exists, it is the unique solution to:
  $$
    V_{\text{opt}}(s) = \max_{a \in \text{Actions}(s)}\left\{ \sum_{s' \in \text{Succ}(s,a)} T(s,a,s')[R(s,a,s
  ') + V_\text{opt}(s'))] \right\}
  $$


  Now let us consider a state $s \in \text{States}$. Then we have:
  \begin{align*}
    V'_{\text{opt}}(s) &= \max_{a \in \text{Actions'}(s)}\left \{ \sum_{s' \in \text{Succ'}(s,a)} T'(s,a,s')[R(s,a,s') + V'_{\text{opt}}(s')] \right\} \tag{definition of $V'_{\text{opt}}$}\\
    &= \max_{a \in \text{Actions}(s)}\left \{ T'(s,a,o)[R'(s,a,o) + V'_{\text{opt}}(o)] + \sum_{s' \in \text{Succ}(s,a)} T'(s,a,s')[R'(s,a,s') + V'_{\text{opt}}(s')] \right\} \tag{$\text{Actions}'(s) = \text{Actions}(s)$ and $\text{Succ}'(s,a) = \{o\} \cup \text{Succ}(s,a)$ by contruction} \\
    &= \max_{a \in \text{Actions}(s)}\left \{ \sum_{s' \in \text{Succ}(s,a)} \gamma T(s,a,s')[R(s,a,s') + V'_{\text{opt}}(s')] \right\} \tag{$R'(s,a,o) + V'_{\text{opt}}(o) = 0$ and definition of $R'$ and $T'$}.
  \end{align*}
  Note that the above equation is precisely the equation that $V_{\text{opt}}(s)$ solves. Therefore, we must have that $V_{\text{opt}}(s)$ and $V'_{\text{opt}}(s)$ are the same function.
\end{enumerate}


\section*{Problem 3}

\begin{enumerate}[label=(\alph*)]
  \item In ``submission.py''.
  \item In ``submission.py''.
\end{enumerate}

\section*{Problem 4}
\begin{enumerate}[label=(\alph*)]
  \item In ``submission.py''.
  \item The policy learned by Q-Learning matches exactly the policy learned by value iteration in the small MDP. However, in the large MDP, the policy learned by Q-Learning differs from the optimal policy significantly (in our experiment, for example, 752 our of 2705 states have non-optimal actions selected by our Q-Learning algorithm). 

  The reason for this is relatively straight-forward. Due to the large nature of the MDP, even after 30,000 trials, our $\epsilon$-greedy learning policy will still leave a significant portion of the state-action space unexplored. This is further exacerbated by the fact that our feature extractor for this section is the identityFeatureExtractor, which treats each state-action pair completely independently. This means that for all of the un-explored state-action space, we actually have learned \textit{nothing} and cannot generalize.

  As such, for this much larger MDP, $Q$-learning performs poorly.
\end{enumerate}
\end{document}